%Seitengroesse, Schriftgroesse und Vorlage (unbedingt noetig)
\documentclass[a4paper, 12pt]{article}

%Benoetigte Pakete werden (zur Uebersicht) in einer anderen Datei geladen
\usepackage[utf8]{inputenc} % Kodierung
\usepackage[english]{babel} % Sprache
\usepackage{graphicx} % immer benötigt für das Einbinden von Graphiken
\usepackage{parskip} % Für den Abstand zwischen 2 Absätzen.
\setlength{\parskip}{12pt plus80pt minus10pt} % Genaue Einstellung von parskip
\usepackage{csquotes} % Für ordentlichen Anführungszeichen
\usepackage[citestyle=numeric-comp,
sorting=none]{biblatex} % bibtex backend für Literaturverzeichnis
\usepackage{xcolor} %Paket für farbige Texte
\usepackage{setspace} %Paket fuer groessere Abstaende, zB fuer das Titelblatt
\addbibresource{literatur/bibliography.bib} % Einbinden der Literatur
\usepackage{url}
\usepackage[activate={true,nocompatibility},
	final,
	tracking=true,
	kerning=true,
	expansion=true,
	spacing=true,
	factor=1050,
	stretch=25,
	shrink=10]{microtype} % Für die Feineinstellung der Zeichensetzung.
\usepackage{booktabs}
\usepackage{multicol}
\usepackage{fancyvrb}
\usepackage[hidelinks]{hyperref} % Klickbare aber nicht markierte Links im PDF
\usepackage{fancyhdr} % Für schönere Kopf-/Fußzeilen und Fußnoten.
\usepackage[right=2.5 cm, left=2.5 cm, top=2.5 cm, bottom=2.5 cm]{geometry} % Seitenränder
\usepackage[allow-number-unit-breaks=true]{siunitx}

%% ADDED BY SNEHADEEP %%
\usepackage{url}
\usepackage{graphicx}
\usepackage{subcaption}
\usepackage{siunitx}
% \sisetup{output-exponent-marker=\ensuremath{\mathrm{e}}}
\usepackage{amsmath}
\usepackage{placeins}
\usepackage{amssymb}
\usepackage{algpseudocode}
\usepackage{algorithm}

\usepackage{ulem}  % Include this package for strikethrough

\newcommand\norm[1]{\lVert#1\rVert}
\DeclareMathOperator{\sign}{sign}

%Gestaltungen (auch in anderer Datei moeglich)
\fancyhf{} %Überschreibt den Kopfzeilen- und Fusszeilenstil der Vorlage
\rfoot{\thepage} %Seitenzahl unten rechts
\renewcommand{\headrulewidth}{0pt} %Loescht die Linie in der Kopfzeile aus der Vorlage

%Hier endet der Vorspann (fuer seitenuebergreifende Entscheidungen)
\begin{document}

%Das Titelblatt wird zur Uebersicht in einer anderen Datei gestaltet
\pagenumbering{gobble}
\newgeometry{margin=2.5cm} %Geaenderte Abstaende zum Rand fuer das Titelblatt
\thispagestyle{empty} %Vorlage komplett ohne Kopf- und Fusszeile
\begin{titlepage}
\begin{center}
	\doublespacing %Doppelter Zeilenabstand
	\textsc{\huge CS6040: Router Architecture and Algorithms}\\
	\vspace{2.0cm} %Vertikaler Abstand
	\onehalfspacing
	\textsc{\Large Semester: Jul-Nov 2024}\\
	% \textsc{\Large Facharbeit}\\
	\vspace{1.0cm}
	
	\rule{\linewidth}{0.5mm}\\ %horizontale Linie
	\vspace{1.4cm}
	\huge \textbf{Assignment 3 Report}\\ %Groesse soll ab jetzt huge sein und der Text fett
	\vspace{1cm}
	%Groesse wird zeitweise auf large gesezt und zurueck auf normal
	\large Packet Switch Queueing \normalsize
	\vspace{0.3cm}
	\rule{\linewidth}{0.5mm}\\
	\vspace{2.4cm}
\end{center}

\onehalfspacing

\begin{minipage}[t]{0.8\textwidth}
	\begin{itemize}
	\item[\emph{Name:}] Snehadeep Gayen
	\item[\emph{Roll:}] CS21B078
	\end{itemize}
\end{minipage}

\vspace{2.9cm}

%flushright = rechtsbuendig, emph = kursiv, today = automatisches Datum
\flushright \emph{Submission Date:} 6th Oct 2024
\end{titlepage}
%Nach der Titelseite soll die geplante Geometrie (z.B. Abstand zum Rand) wieder hergestellt werden
\restoregeometry

\onehalfspacing
\pagestyle{fancy}
\thispagestyle{empty}

% BASIEREND AUF: https://www.overleaf.com/latex/templates/facharbeit-polonaise-rupprecht-gymnasium/cjxzxppjgybx

%%%%%%%%%%%%%%%%%%%%%%%%%%%%%
%%%Ab hier Inhalt einfügen%%%
% %%%%%%%%%%%%%%%%%%%%%%%%%%%%%
% \newpage
% \include{src/01Kurzfassung}
% \newpage

%Inhaltsverzeichnis wird automatisch erzeugt
%Bei Änderungen muss u.U. mehrfach kompiliert werden
% \tableofcontents	

\newpage
\pagenumbering{arabic}
% \setcounter{page}{6} %Erst hier soll die Nummerierung der Seiten beginnen

\input{src/Dataset}

% \newpage
% \section*{Question 2}
% \input{src/Q2}

\end{document}