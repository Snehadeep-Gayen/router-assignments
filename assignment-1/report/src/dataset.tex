
% Begin with describing virtual circuit switching
\begin{section}{Virtual Circuit Switching}

Virtual Circuit Switching is a method of switching in which a dedicated path is established between the source and destination before the actual data transfer begins. This path is called a virtual circuit. The virtual circuit is a logical path that is established between the source and destination. The virtual circuit is established by sending a setup message from the source to the destination. The destination sends back an acknowledgment message to the source. Once the virtual circuit is established, the data transfer begins. The data transfer takes place over the virtual circuit. The virtual circuit is torn down after the data transfer is complete. Virtual Circuit Switching is used in ATM (Asynchronous Transfer Mode) networks.

\end{section}


\begin{section}{Experiment}

    \begin{subsection}{Objective}
        Compare the probability of denial of connection in a virtual circuit switched network in two different topologies under different traffic loads, and constraints (pessimistic and optimistic).

    \end{subsection}


    \begin{subsection}{Topology}
        Two topologies are considered for the experiment:
        \begin{enumerate}
            \item NFSNet Topology : A network with 12 nodes and 15 links.
            \item ARPANet Topology : A network with 20 nodes and 32 links.
        \end{enumerate}
    \end{subsection}

    

\end{section}